\documentclass[12pt]{article}
\usepackage[utf8]{inputenc}
\usepackage{float}
\usepackage{amsmath}
\usepackage{amssymb}
\usepackage{multicol}
\usepackage[shortlabels]{enumitem}
\usepackage{tikz}
\usepackage{graphicx}
\graphicspath{ {imgs/} }
\usepackage{listings}


\usepackage[hmargin=3cm,vmargin=6.0cm]{geometry}
\topmargin=-2cm
\addtolength{\textheight}{6.5cm}
\addtolength{\textwidth}{2.0cm}
\setlength{\oddsidemargin}{0.0cm}
\setlength{\evensidemargin}{0.0cm}

\begin{document}

\section*{Student Information}
Full Name: Berk Ulutaş \\
Id Number: 2522084 \\

\section*{Answer 1}
\begin{itemize}
    \item Alan Turing was born in June 23 1912 and died in June 7 (i) \textbf{1954}.
    \item Turing played a crucial role in breaking the (ii) \textbf{Enigma} code during World War II.
    \item The now-famous (iii) \textbf{Turing Test} , proposed in his paper Computing Machinery and Intelligence (1950), is an attempt to define a standard for a machine to be called “intelligent".
    \item One of his most-cited works, titled (iv) \textbf{"The Chemical Basis of Morphogenesis"} was published in 1952, which proposed a mechanism as to how inhomogeneous patterns in nature arise from symmetric starting states.
    \item The 2014 movie titled (v) \textbf{"The Imitation Game"} aims to give a biographical portrait of Turing.
\end{itemize}

\section*{Answer 2}
\subsection*{a)}
$M = (K, \Sigma, \Delta, q_s, \{h\})$ \\ 
$K = \{q_s, q_0, q_1, q_2, q_{loop}, h\}$ \\
$\Sigma = \{a,b,\rhd, \sqcup\}$ \\ 
$q_s$ is initial state \\
$h$ is halting state \\ 

Assuming initial configuration of $M$ of input $w$ in the form $(q_0, \rhd \underline{\sqcup} w)$
and $\delta$ is given by following table \\

\begin{tabular}{c|c|c}
     $q$ & $\sigma$ & $\delta(q,\sigma)$\\
     \hline
     $q_s$ & $\sqcup$ & $(q_0,\rightarrow)$ \\
     $q_s$ & $\rhd$ & $(q_s,\rightarrow)$ \\
     $q_0$ & $a$ & $(q_0,\rightarrow)$ \\
     $q_0$ & $b$ & $(q_1,\rightarrow)$ \\
     $q_0$ & $\sqcup$ & $(q_{loop},\sqcup)$ \\
     $q_0$ & $\rhd$ & $(q_0,\rightarrow)$ \\
     $q_1$ & $a$ & $(q_1,\rightarrow)$ \\
     $q_1$ & $b$ & $(q_2,\rightarrow)$ \\
     $q_1$ & $\sqcup$ & $(q_{loop},\sqcup)$ \\
     $q_1$ & $\rhd$ & $(q_1,\rightarrow)$ \\
     $q_2$ & $a$ & $(q_{loop},a)$ \\
     $q_2$ & $b$ & $(q_{loop},b)$ \\
     $q_2$ & $\sqcup$ & $(h,\sqcup)$ \\
     $q_2$ & $\rhd$ & $(q_2,\rightarrow)$ \\
     $q_{loop}$ & $a$ & $(q_{loop},a)$ \\
     $q_{loop}$ & $b$ & $(q_{loop},b)$ \\
     $q_{loop}$ & $\sqcup$ & $(q_{loop},\sqcup)$ \\
     $q_{loop}$ & $\rhd$ & $(q_{loop},\rightarrow)$
\end{tabular}

\includegraphics[scale=0.27] {q2a}
\subsection*{b)}
\includegraphics[scale=0.2] {q2b}

\section*{Answer 3}

\begin{itemize}
    \item The Turing machine starts by copying the values of a and b onto the second and third tapes, respectively. The machine starts in the state where it reads from the first tape. If it reads a binary digit, it writes that digit on the second tape and moves right. If it reads a comma, it changes to a new state, where it reads a binary digit, writes on the third tape and moves right. This step ensures, a is copied onto tape 2 and b is copied onto tape 3.
    \item Erase the content on tape 1, and copy tape 2 to tape 1. After this step we need to subtract 1 from b. Use $M_-$ to subtract 1. Update b on tape 3 with the result. (Since we know a and b are positive integers(said in discussion forum), do not need to worry about $b=0$ case.)
    \item Enter a loop that will continue until b (on tape 3) becomes zero:
    \begin{itemize}
        \item Use $M_{\times}$ to multiply the current result on tape 1 (initially, this is a) with a from tape 2. Write the result back to tape 1.
        \item Use $M_-$ to subtract 1 from b on tape 3. Update b on tape 3 with the result.
        \item Check if b has become zero (by looking at tape 3). If it has, exit the loop. If it hasn't, repeat the loop.
    \end{itemize}
    \item When the machine exits the loop (because b has become zero), the computation is done. The first tape now contains the result of $a^b$, and the machine halts.
\end{itemize}


\end{document}

