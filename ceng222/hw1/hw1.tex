\documentclass[12pt]{article}
\usepackage[utf8]{inputenc}
\usepackage{float}
\usepackage{amsmath}


\usepackage[hmargin=3cm,vmargin=6.0cm]{geometry}
%\topmargin=0cm
\topmargin=-2cm
\addtolength{\textheight}{6.5cm}
\addtolength{\textwidth}{2.0cm}
%\setlength{\leftmargin}{-5cm}
\setlength{\oddsidemargin}{0.0cm}
\setlength{\evensidemargin}{0.0cm}

%misc libraries goes here
\usepackage[shortlabels]{enumitem}
\usepackage{listings}
\usepackage{graphicx}
\graphicspath{ {images/} }
\begin{document}

\section*{Student Information } 
%Write your full name and id number between the colon and newline
%Put one empty space character after colon and before newline
Full Name : Berk Ulutaş \\
Id Number : 2522084 \\

% Write your answers below the section tags
\section*{Answer 1}

\subsection*{a)}
We calculate the expected value $E(X) = \sum\limits_{x} xP(x)$. Since the probabilities of all faces on a die are equal, we can write it as:
\begin{itemize}
    \item $E(B) = \dfrac{1}{6} \cdot (1+2+3+4+5+6) = 3.5 $ 
    \item $E(Y) = \dfrac{1}{8} \cdot (1+1+1+3+3+3+4+8) = 3 $ 
    \item $E(R) = \dfrac{1}{10} \cdot (2+2+2+2+2+3+3+4+4+6) = 3 $
\end{itemize}
\subsection*{b)}

To maximize the total value on three dice we need to choose the option which gives higher expected value.
\begin{itemize}
    \item Option 1: Rolling a single dice for each color.  $$E(X)= E(B)+E(Y)+E(R) = 3.5+3+3 =9.5$$
    \item Option 2: Rolling three blue dice.
    $$E(3B)= 3\cdot E(B) = 3 \cdot 3.5=10.5$$
\end{itemize}
Option 2 (Rolling three blue dice.) should be chosen since the expected value is higher. 
\subsection*{c)} 
If it is guaranteed that the yellow die's value will be 8 we can say that $E(Y) = 8$. Then
\begin{itemize}
    \item Option 1: Rolling a single dice for each color.  $$E(X)= E(B)+E(Y)+E(R) = 3.5+8+3 =14.5$$
    \item Option 2: Rolling three blue dice.
    $$E(3B)= 3\cdot E(B) = 3 \cdot 3.5=10.5$$
\end{itemize}
Option 1 (Rolling a single dice for each color.) should be chosen since the expected value is higher. 

\subsection*{d)} 
$R,B,Y$ denotes the events that red, blue, yellow die is picked respectively. The question asks for $P(R|3)$. From Bayes' Rule, we have
$$P(R | 3) = \dfrac{P(3|R) \cdot P(R)}{P(3)}$$

\begin{itemize}
    \item $P(R) = P(B) = P(Y) = \dfrac{1}{3}$ (each color has equal probability)
    \item $P(3|B) = \dfrac{1}{6}$, $P(3|Y) = \dfrac{3}{8}$, $P(3|R) = \dfrac{2}{10}$
    \item $P(3) = \dfrac{1}{3} \cdot P(3|B) + \dfrac{1}{3} \cdot P(3|Y) + \dfrac{1}{3} \cdot P(3|R) = \dfrac{89}{360}$ (using law of total probability, since events are mutually exclusive and exhaustive. Choosing any of dice is $\frac{1}{3}$)
\end{itemize}
$$P(R | 3) = \dfrac{\dfrac{2}{10} \cdot \dfrac{1}{3}}{\dfrac{89}{360}} = \dfrac{24}{89} \approx 0.2697$$
\subsection*{e)} 
To get total value of 5, we have following combinations: 
\begin{itemize}
    \item Blue die: 1, Yellow die 4 
    \item Blue die: 2, Yellow die 3 
    \item Blue die: 4, Yellow die 1 
\end{itemize}
Now, calculate the probability of each of these combinations. (Since events are independent we can simply multiply to find probability both events happening together): 
\begin{itemize}
    \item $P\{\text{Blue die: 1}\} \cap P\{\text{Yellow die: 4}\}$
    \begin{itemize}
        \item $P\{\text{Blue die: 1}\}  = \frac{1}{6}, P\{\text{Yellow die: 4}\}= \frac{1}{8} $
        \item $\dfrac{1}{6} \cdot \dfrac{1}{8} = \dfrac{1}{48}$
    \end{itemize}
    \item $P\{\text{Blue die: 2}\} \cap P\{\text{Yellow die: 3}\}$
    \begin{itemize}
        \item $P\{\text{Blue die: 2}\}  = \frac{1}{6}, P\{\text{Yellow die: 3}\}= \frac{3}{8} $
        \item $\dfrac{1}{6} \cdot \dfrac{3}{8} = \dfrac{3}{48}$
    \end{itemize}
    \item $P\{\text{Blue die: 4}\} \cap P\{\text{Yellow die: 1}\}$
    \begin{itemize}
        \item $P\{\text{Blue die: 4}\}  = \frac{1}{6}, P\{\text{Yellow die: 1}\}= \frac{3}{8} $
        \item $\dfrac{1}{6} \cdot \dfrac{1}{8} = \dfrac{3}{48}$
    \end{itemize}
\end{itemize}
Finally, we need to add up the probability of all events happening to get overall probability
$$\dfrac{1}{48} + \dfrac{3}{48} + \dfrac{3}{48} = \dfrac{7}{48}$$
\section*{Answer 2}

\subsection*{a)}
We can think this question as binomial distribution with parameters $n = 80$ and $p = 0.025$ . Let X be denote the distributors of company A will offer a discount tomorrow.
\begin{itemize}
    \item  We try to find $P(X \geq 4)$
    \item $P(X \geq 4) = 1 - P(X < 4)$
    \item We can calculate $1 - (P(X=0) + P(X=1) + P(X=2) + P(X=3))$
    \item $1-(\binom{80}{0} \cdot 0.025^{0} \cdot 0.975^{80} + \binom{80}{1} \cdot 0.025^{1} \cdot 0.975^{79} + \binom{80}{2} \cdot 0.025^{2} \cdot 0.975^{78} + \binom{80}{3} \cdot 0.025^{3} \cdot 0.975^{77})$
    \item or we can use octave online to calculation with this query \verb|1-binocdf(3,80,0.025)|
\end{itemize}
\begin{equation*}
    \begin{split}
        P(X \geq 4) &= 1 - P(X < 4) \\
                    &= \verb |1-binocdf(3,80,0.025) \\
                    &= 0.1406  
    \end{split}
\end{equation*}

\subsection*{b)} 
To calculate the probability of getting a discount on a phone within two days, we can use the binomial distribution. Let X and Y be denote, the distributors of company A and B will offer a discount tomorrow, respectively.
\begin{itemize}
    \item First we need to calculate not getting discount on a specific day for each company
    \item $P(X = 0) = \binom{80}{0}(0.025)^0(1-0.025)^{80} = 0.975^{80}$ 
    \item $P(Y = 0) = \binom{1}{0}(0.1)^0(1-0.1)^{1} = 0.9^{1}$ 
    \item Now, we calculate the probability of not getting a discount from both companies on a specific day. Since companies act independently we can simply multiply what we found: $$0.975^{80} \cdot 0.9^{1} = 0.1187$$
    \item Next, we find the probability of not getting a discount from both companies in two days:
    $$0.1187^{2}= 0.0141$$
    \item Now, we calculate the probability of getting a discount within two days by subtracting the above probability from 1:
    $$1-0.0141 = 0.9859$$
    \item So, the probability of being able to buy a phone within two days is $\approx 0.986$
\end{itemize}


\section*{Answer 3}

\begin{verbatim}
    blue = [1 2 3 4 5 6];
    yellow = [1 1 1 3 3 3 4 8];
    red = [2 2 2 2 2 3 3 4 4 6];
    
    num_iterations = 5000;
    total_value_option_1 = 0;
    total_value_option_2 = 0;
    num_cases_option_2_greater = 0;
    
    for i = 1:num_iterations
        dice_11 = red(randi(length(red)));
        dice_12 = yellow(randi(length(yellow)));
        dice_13 = blue(randi(length(blue)));
        
        dice_21 = blue(randi(length(blue)));
        dice_22 = blue(randi(length(blue)));
        dice_23 = blue(randi(length(blue)));
    
        total_value_option_1 += dice_11 + dice_12 + dice_13;
        total_value_option_2 += dice_21 + dice_22 + dice_23;
    
        if (dice_21 + dice_22 + dice_23) > (dice_11 + dice_12 + dice_13)
            num_cases_option_2_greater++;
        end
    end
    
    avg_total_value_option_1 = total_value_option_1 / num_iterations
    avg_total_value_option_2 = total_value_option_2 / num_iterations
    percent_option_2_greater = num_cases_option_2_greater / num_iterations * 100
    
\end{verbatim} 
\includegraphics[width=\textwidth]{octave-ss} \\ \\

Based on the given results of 5000 tries, we can see that the average total value for Option 2 (10.499) is indeed higher than the average total value for Option 1 (9.537). Analysis results match the expected values, it means we did the analysis correctly. Additionally, the percentage of times where Option 2 results in a higher total value (56.2\%) supports the conclusion that Option 2 is the better strategy for maximizing the total value.

\end{document}
