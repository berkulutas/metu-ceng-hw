\documentclass[12pt]{article}
\usepackage[utf8]{inputenc}
\usepackage{float}
\usepackage{amsmath}



\usepackage[hmargin=3cm,vmargin=6.0cm]{geometry}
%\topmargin=0cm
\topmargin=-2cm
\addtolength{\textheight}{6.5cm}
\addtolength{\textwidth}{2.0cm}
%\setlength{\leftmargin}{-5cm}
\setlength{\oddsidemargin}{0.0cm}
\setlength{\evensidemargin}{0.0cm}

%misc libraries goes here
\usepackage[shortlabels]{enumitem}
\usepackage{listings}
\usepackage{graphicx}
\graphicspath{ {images/} }
\usepackage{multicol}
\usepackage{matlab-prettifier}


\begin{document}

\section*{Student Information } 
%Write your full name and id number between the colon and newline
%Put one empty space character after colon and before newline
Full Name : Berk Ulutaş \\
Id Number : 2522084 \\


% Write your answers below the section tags
\section*{Answer 1}

\subsection*{a)} 
We can use formula 9.9 from our textbook page 259 since we have small sample and we don't know population standard deviation.\\
$$CI = \bar{X} \pm t_{\alpha/2}\frac{s}{\sqrt{n}}$$
Assuming Normal distribution of these consumptions.
\begin{itemize}
    \item The sample size $n= 16$
    \item Confidence $1-\alpha = 0.98$ then $\alpha = 0.02$
    \item The sample mean consumption is $\Bar{X} = \frac{8.4+7.8+\dots+8.5}{16} \approx 6.81$
    \item The sample standard deviation is $s = \sqrt{\frac{\Sigma (x_i - \bar{x})^2}{n-1}} = \frac{(8.4-6.81)^2 + (7.8-6.81)^2 + \dots + (8.5-6.81)^2}{16-1} \approx 1.06 $
    \item The critical value of $t$ distribution with $n-1 = 15$ degrees of freedom is $t_{\alpha/2} = t_{0.01} = 2.602$ (using appendix table A5 from textbook )
\end{itemize}
$$CI = 6.81 \pm 2.602\cdot\dfrac{1.06}{\sqrt{16}}$$ 
$$= [6.12, 7.50]$$

\subsection*{b)}
To determine if the improvement in the engine is effective, we need to set up the null and alternative hypotheses and conduct a hypothesis test.
\begin{itemize}
    \item $H_0$ : The improvement in the engine did not result in a significant reduction in the gasoline consumption. ($\mu = 7.5$)
    \item $H_A$ : The improvement in the engine resulted in a significant reduction in the gasoline consumption. ($\mu < 7.5$)
    \item We will conduct a one-tailed hypothesis test using the given significance level $\alpha = 0.05$ 
\end{itemize}
To perform hypothesis test we can use t test formula from the textbook page 276, since we have the sample mean, sample standard deviation, and sample size.
$$t= \dfrac{\bar{X}-\mu_0}{s/\sqrt{n}}$$
We have: 
\begin{itemize}
    \item $\bar{X} = 6.81$ (sample mean)
    \item $\mu_0 = 7.5$ (population mean under null hypothesis)
    \item $s= 1.06$ (sample standard deviation)
    \item $n=16$ (sample size)
\end{itemize}
Compute the T-statistic:
$$t= \dfrac{6.81 - 7.5}{1.06/\sqrt{16}} = -2.60$$

The rejection region $\mathcal{R} = (-\infty, -1.75]$, where we used T-distribution with $16-1= 15$ degree of freedom and $\alpha=0.05$ since we used left tail test ($t_\alpha = \text{tinv}(0.05, 15) = -1.75$ computed with octave online)

Since $t \in \mathcal{R}$, we reject the null hypothesis and conclude that there is significant evidence of the improvement resulted reduction in the gasoline consumption.

\subsection*{c)} 
If we assume car was consuming 6.5 liters of gasoline per 100 km. Our null and alternative hypothesis are followings:
\begin{itemize}
    \item $H_0$ : $\mu = 6.5$
    \item $H_A$ : $\mu < 6.5$
\end{itemize}

Alternative $H_A$ : $\mu < 6.5$ covering the region to the left of $H_0$ is one-sided, left-tail. Since it is left tail, the rejection are in negative side.\\

Let's analyze the the test statistic
$$t = \dfrac{6.81 - 6.5}{1.06/\sqrt{16}}$$
\begin{itemize}
    \item Numerator $(6.81 - 6.5 > 0)$ and Denominator $(1.06/\sqrt{16} > 0)$
    \item Without any statistic test calculation we can easily see that $t>0$
    \item We said that our rejection area in negative side and we know that $t>0$. So, $t \notin \mathcal{R}$
    \item Since we know our test statistic is not in rejection region we can immediately accept $H_0$ without any statistic test calculation.
\end{itemize}


\section*{Answer 2}

\subsection*{a)}
\begin{itemize}
    \item $H_0$ : $\mu = 5000$ (the prices are similar to those of the last year.)
    \item $H_A$ : $\mu > 5000$ (there is an increase in the prices.)
\end{itemize}
Ali's claim should be considered as null hypothesis.

\subsection*{b)} 
We test the null hypothesis $H_0$ : $\mu = 5000$ against a one sided right-tail alternative $H_A$ : $\mu > 5000$, since we only interested to know if the mean number of rent prices has increased with significance level $\alpha = 0.05$

To perform hypothesis test we can use z test formula from the textbook page 273, since we have one sample and we know population mean and standard deviation.
$$z = \dfrac{\bar{X}-\mu_0}{s/\sqrt{n}}$$

We have: 
\begin{itemize}
    \item $\bar{X} = 5500$ (sample mean)
    \item $\mu_0 = 5000$ (mean)
    \item $s= 2000$ (standard deviation)
    \item $n=100$ (sample size)
\end{itemize}

$$z = \dfrac{5500-5000}{2000/\sqrt{100}} = 2.5$$
The critical value is (\texttt{}{norminv(0.05)} computed with octave online and normal distribution is symmetric): 
$$z_\alpha  = z_{0.05} = 1.645$$
With right-tail alternative the rejection region $\mathcal{R} = [1.645, \infty)$. 


Since our test statistic $z =2.5 \in \mathcal{R}$, we reject the null hypothesis. The data provided sufficient evidence in favor of the alternative hypothesis. So, Ahmet can claim that there is an increase in the rent prices compared to the last year at a 5\% level of significance.

\subsection*{c)}
In part b we have found at 5\% level of significance Ahmet can claim that increase in the rent prices compared to the last year. Let us compute the P-value. 
\begin{itemize}
    \item We have found Z-statistic in part b $$Z_{obs}= 2.5$$
    \item To compute P-value we can use formula from Table 9.3 page 283 in our textbook.
    \item We find that the P-value for the right-tail alternative is: ($\Phi(2.5) = 0.9938$ taken from textbook appendix table A4 )
    $$P = \mathbf{P}\{Z \geq Z_{obs}\} = \mathbf{P}\{Z \geq 2.5\}= 1 - \Phi(2.5) = 0.0062$$
\end{itemize}

The P-value is very low, a low P-value indicates that such an extreme test statistic is unlikely if $H_0$ is true. Therefore, P-value supports that Ahmet can reject the null hypothesis.

\subsection*{d)}
Let's start with determining null and alternative hypothesis. Subscript x represents Ankara a and subscript y represents Istanbul.
\begin{itemize}
    \item $H_0$ : $\mu_x = \mu_y$ (the prices in Ankara is equal to  the prices in Istanbul)
    \item $H_A$ : $\mu_x < \mu_y$ (the prices in Ankara is less than  the prices in Istanbul)
\end{itemize}

We test the null hypothesis $H_0$ : $\mu_x = \mu_y$ against one sided left-tail alternative $H_A$ : $\mu_x < \mu_y$ with significance level $\alpha = 0.01$

We have two-sample and to perform hypothesis test we can use z test formula from the textbook page 273.

$$z =\dfrac{\Bar{X} - \Bar{Y} - D}{\sqrt{\dfrac{\sigma_X^2}{n} + \dfrac{\sigma_Y^2}{m} }}$$

We have:
\begin{multicols}{2}
    \begin{itemize}
        \item $\bar{X} = 5500$
        \item $\sigma_X = 2000$
        \item $n = 100 $
        \item $\bar{Y} = 6500$
        \item $\sigma_Y = 3000$
        \item $m = 60 $
    \end{itemize}
\end{multicols}
The tested value is $D = 0 $. The test statistic the equals
$$z = \dfrac{5500 - 6500}{\sqrt{\frac{2000^2}{100} + \frac{3000^2}{60}}} = -2.29$$

Let's calculate acceptance and rejection regions. This is a left tail test. The critical value is:
$$z_\alpha = z_{0.01} = -2.326$$

With left-tail alternative the rejection region $\mathcal{R} = (-\infty, -2.326]$


Since our test statistic $z = -2.29 \notin \mathcal{R}$, we cannot reject $H_0$. The evidence against $H_0$ is insufficient. They cannot claim that the prices in Ankara is lower than the prices in Istanbul.

\section*{Answer 3}
Let's start with determining null and alternative hypothesis
\begin{itemize}
    \item $H_0$ : number of rainy days in Ankara is independent of the season
    \item $H_A$ : number of rainy days in Ankara is dependent of the season
\end{itemize}

Here is the obtained data: \\
\begin{center}
    \begin{tabular}{c|c c c c|c}
    Obtained  & Winter & Spring & Summer & Autumn & Total \\ \hline
    Rainy     & 34     & 32     & 15     & 19     & 100       \\
    Non-rainy & 56     & 58     & 75     & 71     & 260       \\ \hline
    Total & 90     & 90     & 90     & 90     & 360      
    \end{tabular}
\end{center}

Testing independence, we compute the estimated expected counts with this formula which is given in our textbook page 312:
$$\widehat{Exp}(i,j) = \dfrac{(n_{i \cdot })(n_{ \cdot j})}{n}$$
\begin{itemize}
    \item $\widehat{Exp}(1,j) = \frac{90\cdot 100}{360} = 25 $ for $j = 1,2,3,4$
    \item $\widehat{Exp}(2,j) = \frac{90\cdot 260}{360} = 65 $ for $j = 1,2,3,4$
\end{itemize}

\begin{center}
    \begin{tabular}{c|cccc|c}
        Expected  & Winter & Spring & Summer & Autumn & Total \\ \hline
        Rainy     & 25     & 25     & 25     & 25     & 100       \\
        Non-rainy & 65     & 65     & 65     & 65     & 260       \\ \hline
        Total & 90     & 90     & 90     & 90     & 360      
    \end{tabular}
\end{center}

Then we can use the formula from our textbook page 312 for calculating test statistic:
$$\chi_{obs}^2 = \sum_{i=1}^k \sum_{j=1}^m \dfrac{\{Obs(i,j) - \widehat{Exp}(i,j)\}^2}{\widehat{Exp}(i,j)}$$

The test statistic is: 
$$\chi_{obs}^2 = \frac{(34-25)^2}{25} + \dots + \frac{(71-65)^2}{65} = 14.73$$
and it has $(2-1)\cdot (4-1) = 3$ degrees of freedom.

We can calculate p value using octave online:
$$\text{P} = 1- chi2cdf(14.73, 3) \approx 0.002$$
According to table in our textbook page 282. Practically we can reject $H_0$ if $P < 0.01$. 
\\


Based on our calculation, the obtained p-value of 0.002 is indeed less than 0.01. Therefore, we have significant evidence to reject the null hypothesis, which suggests that there is no relationship between the number of rainy days in Ankara and the season. Instead, the evidence supports the alternative hypothesis, indicating that the number of rainy days in Ankara is dependent on the season


\section*{Answer 4}
\begin{lstlisting}[style=Matlab-editor]
pkg load statistics

% TO TEST WITH DIFFERENT INPUT CHANGE X
X = [34 32 15 19; 56 58 75 71; ]

Row = sum(X')'; Col = sum(X); Tot = sum(Row);
k = length(Col); m = length(Row);
e= zeros(size(X));

for i =1:m;
    for j=1:k;
        e(i,j) = Row(i)*Col(j)/Tot; 
    end
end

chisq = ((X-e).^2)./e;
chi_sq_obs = sum(sum(chisq));
Pval= 1-chi2cdf(chi_sq_obs, (k-1)*(m-1));

fprintf("chi_sq_obs = %g\n", chi_sq_obs);
fprintf("P-val = %g\n", Pval);


\end{lstlisting}

\includegraphics[width=\textwidth] {images/q4.png}






\end{document}
