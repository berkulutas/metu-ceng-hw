\documentclass[12pt]{article}
\usepackage[utf8]{inputenc}
\usepackage{float}
\usepackage{amsmath}
\usepackage{tikz-cd}
\usepackage[shortlabels]{enumitem}

\usepackage[hmargin=3cm,vmargin=6.0cm]{geometry}
%\topmargin=0cm
\topmargin=-2cm
\addtolength{\textheight}{6.5cm}
\addtolength{\textwidth}{2.0cm}
%\setlength{\leftmargin}{-5cm}
\setlength{\oddsidemargin}{0.0cm}
\setlength{\evensidemargin}{0.0cm}

\begin{document}

\section*{Student Information } 
%Write your full name and id number between the colon and newline
%Put one empty space character after colon and before newline
Full Name : Berk Ulutaş \\
Id Number : 2522084 \\

% Write your answers below the section tags
\section*{Answer 1}

Let $G(x)$ be the generating function for the sequence $\{a_n\}$. $$G(x) = \sum_{n=0}^\infty a_n x^n$$

For this recurrence relation $a_n = 3a_{n-1} + 4a_{n-2} $ multiply each term by $x^n$ and sum each term over all positive $n \geq 2 $
\begin{align*}
    \sum_{n=2}^\infty a_n x^n &= \sum_{n=2}^\infty 3 a_{n-1} x^n + \sum_{n=2}^\infty 4a_{n-2}x^n \\
    \sum_{n=2}^\infty a_n x^n + a_0 + a_1x -(a_0 + a_1x) &= 3x\sum_{n=2}^\infty a_{n-1} x^{n-1} + 4x^2\sum_{n=2}^\infty a_{n-2}x^{n-2} \\
    \sum_{n=0}^\infty a_n x^n -(a_0 + a_1x) &= 3x\sum_{n=2}^\infty a_{n-1} x^{n-1} + 4x^2\sum_{n=2}^\infty a_{n-2}x^{n-2} \\
    \sum_{n=0}^\infty a_n x^n -(a_0 + a_1x) &= 3x(\sum_{n=2}^\infty a_{n-1} x^{n-1} + a_0 - a_0) + 4x^2\sum_{n=2}^\infty a_{n-2}x^{n-2} \\ 
    \sum_{n=0}^\infty a_n x^n -(a_0 + a_1x) &= 3x(\sum_{n=1}^\infty a_{n-1} x^{n-1} - a_0) + 4x^2\sum_{n=2}^\infty a_{n-2}x^{n-2} \\
    \sum_{n=0}^\infty a_n x^n -(a_0 + a_1x) &= 3x(\sum_{n=0}^\infty a_{n} x^{n} - a_0) + 4x^2\sum_{n=0}^\infty a_{n}x^{n} \\
\end{align*}
Right now we have
$$G(x) - a_0 - a_1x = 3x(G(x) - a_0) + 4x^2G(x)$$
We have given $a_0  = a_1 = 1$ 
$$G(x) - 1 - x = 3x(G(x) - 1) + 4x^2G(x)$$
$$G(x)(1-3x-4x^2) = 1-2x$$
$$G(x)((1+x)\cdot(1-4x)) = 1-2x$$
Rewrite the equation to solve $G(x)$, and we have
$$G(x) = \dfrac{1-2x}{(1-4x)\cdot(1+x)}$$
Expanding the right hand side of this equation into partial fractions
$$\dfrac{(1-2x)}{(1-4x)\cdot(1+x)} = \dfrac{A}{(1-4x)} + \dfrac{B}{(1+x)}$$
$$1-2x = A(1+x) + B(1-4x)$$
$$ A + B = 1$$
$$ A - 4B = -2$$
We have found that
$$ A = \frac{2}{5}, B = \frac{3}{5} $$
Substitute A and B
$$G(x) = \dfrac{2}{5} \cdot \dfrac{1}{1-4x} + \dfrac{3}{5} \cdot \dfrac{1}{1+x}$$
Using the identity $\dfrac{1}{1-ax} = \sum_{k=0}^\infty a^kx^k$
$$G(x) = \dfrac{2}{5} \sum_{n=0}^\infty  4^nx^n + \dfrac{3}{5} \sum_{n=0}^\infty  (-1)^nx^n$$
$$G(x) = \sum_{n=0}^\infty (\dfrac{2}{5} 4^n + \dfrac{3}{5} (-1)^n)x^n$$
$$a_n = \dfrac{2}{5} 4^n + \dfrac{3}{5} (-1)^n $$

\section*{Answer 2}
\subsection*{a) }
First obtain
$$f(x) = 2 + 5x + 11x^2 + 29x^3 + 83x^4 + 245x^5  +\dots $$
Adding the following two series, we can get $f(x)$.
\begin{align}
    2 + 2x + 2x^2 + 2x^3 + 2x^4 +\dots \\
    0 + 3x + 9x^2 + 27x^3 + 81x^4 \dots 
\end{align}
We can write (1) and (2) as followings 
\setcounter{equation}{0}
\begin{align}
    2(1+x+x^2+x^3+x^4+\dots) &= 2 \sum_{k=0}^\infty x^k \\ 
    3x(1+3x+9x^2+27x^3+ \dots) &= 3x \sum_{k=0}^\infty 3^k x^k
\end{align}
$$f(x) = 2 \sum_{k=0}^\infty x^k + 3x \sum_{k=0}^\infty 3^k x^k$$
Using the identities $\dfrac{1}{1-x} = \sum_{k=0}^\infty x^k$ and $\dfrac{1}{1-ax} = \sum_{k=0}^\infty a^kx^k$
$$f(x) = 2 \cdot \dfrac{1}{1-x} + 3x \cdot \dfrac{1}{1-3x}$$
$$f(x) = \dfrac{-3x^2 -3x +2}{3x^2-4x+1}$$


\subsection*{b) }
$$G(x) = \dfrac{7-9x}{(1-2x)\cdot(1-x)}$$
Expanding the right hand side of this equation into partial fractions

$$\dfrac{7-9x}{(1-2x)\cdot(1-x)} = \dfrac{A}{1-2x} + \dfrac{B}{1-x} $$

$$7-9x = A(1-x) + B(1-2x) $$
$$7-9x = A + B + x(-A-2B) $$
$$A + B = 7$$
$$-A-2B = -9$$
$$A=5, B=2$$
Substitute A and B
$$G(x) = \dfrac{5}{1-2x} + \dfrac{2}{1-x}$$
Using the identity $\dfrac{1}{1-ax} = \sum_{k=0}^\infty a^kx^k$
$$G(x) = 5\sum_{n=0}^\infty 2^nx^n + 2\sum_{n=0}^\infty x^n$$
$$G(x) = \sum_{n=0}^\infty (5\cdot2^n + 2) x^n$$
$$a_n = 5\cdot2^n + 2$$
$$<7, 12, 22, 42, 82, 162,322 \dots>$$

\section*{Answer 3}
\subsection*{a) }
No, R is not an equivalence relation since transitivity property is not satisfied. A relation R on a set is called transitive if whenever $aRb$ and $bRc$ then $aRc$\\ \\
To see this, consider the following counter example. 
\begin{itemize}
    \item let $a=6, b=8$ then there exists a right triangle with sides $a=6, b=8 $ and hypotenuse $p=10$ by Pythagorean theorem
    \item let $b=8, c=12$ then there exists a right triangle with sides $b=8, c=12$ and hypotenuse $q=15$ by Pythagorean theorem
    \item since $a=6, c=12$ and hypotenuse $r=\sqrt{6^2+12^2} = 6\sqrt{5}$ and $r. \notin Z$. Transitivity property is not satisfied
\end{itemize}
\subsection*{b) }
Yes, it is an equivalence relation since it is reflexive, symmetric, and transitive.

\begin{enumerate}
    \item Reflexive:
    \begin{itemize}
        \item Assume $x_1, y_1 \in R$. Then $(x_1, y_1)R(x_1, y_1)$ is true since $2x_1 + y_1 = 2x_1 + y_1 $
    \end{itemize}
    \item Symmetric:
    \begin{itemize}
        \item Assume $x_1, y_1 \in R$ and $x_2, y_2 \in R$
        \item Suppose $(x_1, y_1)R(x_2, y_2)$. Then $2x_1 + y_1 = 2x_2 + y_2$
        \item Then, $(x_2, y_2)R(x_1, y_1)$ is true since $2x_2 + y_2= 2x_1 + y_1$
    \end{itemize}
    \item Transitive:
    \begin{itemize}
        \item Let $x_1, x_2, x_3 \in R$ and $y_1,y_2,y_3 \in R$
        \item Suppose $(x_1, y_1)R(x_2, y_2)$ and $(x_2, y_2)R(x_3, y_3)$
        \item Then we have $2x_1 + y_1 = 2x_2 +y_2$ and $2x_2 + y_2 = 2x_3 +y_3$
        \item We can write $(x_1, y_1)R(x_3, y_3)$ since $2x_1 + y_1 = 2x_2 + y_2 = 2x_3 +y_3$
    \end{itemize}
\end{enumerate}
Equivalence class of $(1,-2)$ is  $[(1,-2)]_{R}$ = $\{ x, y  | 2x-y =0\}$ \\
It represents a line passing through the origin and having a slope of -2.

\section*{Answer 4}
\begin{enumerate} [a)]
    \item Hasse diagram of R 
    \begin{center}
        \begin{tikzcd}
          60 \\
          10 \arrow[u]&18  \\
          2 \arrow[u] \arrow[ur]& 5 \arrow[ul] \\
        \end{tikzcd}  
    \end{center}
    \item Matrix representation for R
    \begin{center}
        $M_{R} = $
        \begin{bmatrix}
        1 & 0 & 1 & 1 & 1 \\
        0 & 1 & 1 & 0 & 1\\
        0 & 0 & 1 & 0 & 1\\
        0 & 0 & 0 & 1 & 0\\
        0 & 0 & 0 & 0 & 1\\
        \end{bmatrix}
    \end{center}
    \item
    \begin{itemize}
        \item $R_{s} = R \cup R^{-1}$
        \item $R = \{(a,b)| \textit{a divides b}\}$ and $R^{-1} = \{(a,b) | \textit{b divides a }\}$
        \item $R_{s} = \{ (a,b) | \textit{a divides b or b divides a}\}$
    \end{itemize}
    Matrix representation for $R_{s}$
    \begin{center}
        $M_{R_{s}} = $
        \begin{bmatrix}
        1 & 0 & 1 & 1 & 1 \\
        0 & 1 & 1 & 0 & 1\\
        1 & 1 & 1 & 0 & 1\\
        1 & 0 & 0 & 1 & 0\\
        1 & 1 & 1 & 0 & 1\\
        \end{bmatrix}
    \end{center}
    All elements $(x, y)$ where $(x, y) \in R_s$ and $(x, y) \notin R$
    $$\{(10,2), (10,5), (18,2), (60,2), (60,5), (60,10)\}$$
    \item
    To create a total ordering that includes all elements of A we need reflexivity, antisymmetry, transitivity and totality properties be satisfied. Since (2,5), (5,18), (18,60) are not related we need to remove from those elements to satisfy total ordering conditions. But it is not possible to remove an element and add an element to provide these situations. Since if we remove one of the elements of the listed pairs above, still there exists a not related pair. On the other hand, if we remove two elements and add one we can satisfy these conditions. For example if we remove 5 and 18. Add 1 we can satisfy all total ordering properties.
\end{enumerate}


\end{document}
