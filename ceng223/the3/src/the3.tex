\documentclass[12pt]{article}
\usepackage[utf8]{inputenc}
\usepackage{float}
\usepackage{amsmath}
\usepackage{amssymb}
\usepackage{multicol}
\usepackage[shortlabels]{enumitem}


\usepackage[hmargin=3cm,vmargin=6.0cm]{geometry}
%\topmargin=0cm
\topmargin=-2cm
\addtolength{\textheight}{6.5cm}
\addtolength{\textwidth}{2.0cm}
%\setlength{\leftmargin}{-5cm}
\setlength{\oddsidemargin}{0.0cm}
\setlength{\evensidemargin}{0.0cm}

%misc libraries goes here

\begin{document}

\section*{Student Information }
%Write your full name and id number between the colon and newline
%Put one empty space character after colon and before newline
Full Name : Berk Ulutaş \\
Id Number :  2522084 \\

% Write your answers below the section tags
\section*{Answer 1}

To construct the proof let $P(n)$ be the proposition "$6^{2n}-1$ is divisible by both 5 and 7. For $n \in N^{+}$"
\noindent 

\begin{enumerate}[1)]

    \item 
        \subsection*{Basis Step}
           We must show the statement is true for $n=1$. The statement $P(1)$ is true because $6^{2} -1 = 35 $ and it is divisible by both 5 and 7. \\
           
    \item
        \subsection*{Inductive Step}
            For the inductive hypothesis we assume that $P(k)$ is true; that is we assume that $6^{2k}-1$ is divisible by both 5 and 7. For arbitrary $k \in N^{+}$ \\
            We must show $P(k+1) = 6^{2 \cdot (k+1)} -1$ is divisible by both 5 and 7.
            \begin{align*}
                P(k+1) &= 6^{2 \cdot (k+1)} -1 \\
                &= 6^{2k+2} -1 \\
                &= 36\cdot6^{2k} -1 \\
                &= (35\cdot6^{2k}) + (6^{2k} -1) \\
            \end{align*}


            clearly the first term $35 \cdot 6^{2k}$ is divisible by both 5 and 7. Using the inductive hypothesis, we conclude that the second term $6^{2k}-1$ is divisible by both 5 and 7. This completes the inductive step.\\

            Hence, $6^{2n}-1$ is divisible by both 5 and 7 by induction.
            
           
\end{enumerate}{}
\pagebreak


\section*{Answer 2}
\noindent 

\begin{enumerate}[1)]
    \item 
        \subsection*{Basis Step}
            We must show the statement is true for $ n = 0$ which is the smallest value of n for this inequality. \\
            $H_{n} \leq 9^n $ must be satisfied for $ n= 0$ \\
            $H_{0} \leq 9^0 $ is true since $H_{0}$ is given as 1. Same as $ H_{1}$ and $H_{2}$\\
            $H_{1} = 5\leq 9^1$ \\
            $H_{2} = 7\leq 9^2$
    \item
        \subsection*{Inductive Step}
            Assume that $H_{j} \leq 9^j$ for $ 0 \leq j \leq k $ where $ k \geq 2$ We must show that $H_{k+1} \leq 9^{k+1}$ to prove it is true for all values of n. 
            $$H_{k+1} = 8H_{k} + 8H_{k-1} + 9H_{k-2}$$
            
            since we assumed $H_{j}$ is true where $0 \leq j \leq k$ , $H_{k} \leq 9^{k}, H_{k-1} \leq 9^{k-1} , H_{k-2} \leq 9^{k-2}$ we can write this inequality\\
            $$8H_{k} + 8H_{k-1} + 9H_{k-2}\leq 8 \cdot 9^{k} + 8 \cdot 9^{k-1} + 9 \cdot 9^{k-2} $$
            $$ 8H_{k} + 8H_{k-1} + 9H_{k-2}\leq 8 \cdot 9^{k} + 8 \cdot 9^{k-1} + 9^{k-1}$$
            $$ 8H_{k} + 8H_{k-1} + 9H_{k-2}\leq 8 \cdot 9^{k} + 9 \cdot 9^{k-1} $$
            $$ 8H_{k} + 8H_{k-1} + 9H_{k-2}\leq 8 \cdot 9^{k} +  9^{k} $$
            $$ 8H_{k} + 8H_{k-1} + 9H_{k-2}\leq 9 \cdot 9^{k} $$
            $$ 8H_{k} + 8H_{k-1} + 9H_{k-2}\leq 9^{k+1} $$
            $$ H_{k+1} \leq 9^{k+1}$$
            
            Hence, $H_n \leq 9^n$, for all $n \in  N$ by induction.
\end{enumerate}{}
\pagebreak

\section*{Answer 3}
To construct solution we can group possible arrangements according to $k = $ first zero position of consecutive four zero part. 

\begin{itemize}
    \item For $ k =1 $ we have $2^4$ strings (0000****)
    \item For $ k =2 $ we have $2^3$ strings (10000***)
    \item For $ k =3 $ we have $2^3$ strings (*10000**)
    \item For $ k =4 $ we have $2^3$ strings (**10000*)
    \item For $ k =5 $ we have $2^3$ strings (***10000)
\end{itemize}
This grouping gives us to the number of strings with at least 4 consecutive zeros is $ 2^4 + 4\cdot 2^3 = 48$. By symmetry, the number of strings with at least 4 consecutive ones is the same. However, in this case we would have counted 11110000 and 00001111 twice. \\ \\Hence the total number is $ 48 + 48 - 2 = 94$

\pagebreak

\section*{Answer 4}
We need to choose 1 star from 10 distinct stars, 2 habitable planet from 20 distinct habitable planets and 8 non-habitable planet from 80 distinct habitable planets
\begin{itemize}
    \item 1 star from 10 distinct stars $C(10,1) = \dfrac{10!}{1!\cdot9!}$
    \item 2 habitable planet from 20 distinct habitable planets $C(20,2) = \dfrac{20!}{2!\cdot18!}$
    \item 8 non-habitable planet from 80 distinct non-habitable planets $C(80,8) = \dfrac{80!}{8!\cdot72!}$
\end{itemize}
We can choose $\dfrac{10!}{1!\cdot9!} \cdot \dfrac{20!}{2!\cdot18!} \cdot \dfrac{80!}{8!\cdot72!}$ different star and planet sets by product rule. \\ \\
Now we need to arrange planets around the star such that there are at least 6 non-habitable planets between 2 habitable planets. There can be 6, 7 or 8 planets between 2 habitable planets \\\\Let $H_i$ represents habitable planet and $*$ represents non-habitable planets.
\begin{itemize}
    \item 6 non-habitable planets between 2 habitable planets. There can be 3 different arrangement.
    \begin{itemize}
        \item $H_1, *, *, *, *, *, *, H_2, *, *$
        \item $*, H_1, *, *, *, *, *, *, H_2, *$
        \item $*, *,H_1, *, *, *, *, *, *, H_2$
    \end{itemize}
    \item 7 non-habitable planets between 2 habitable planets.There can be 2 different arrangement.
    \begin{itemize}
        \item $H_1, *, *, *, *, *, *, *, H_2, *$
        \item $*, H_1, *, *, *, *, *, *, *, H_2$
    \end{itemize}
    \item 8 non-habitable planets between 2 habitable planets. There can be 1 different arrangement.
    \begin{itemize}
        \item $H_1, *, *, *, *, *, *, *, *, H_2$
    \end{itemize}
    \item We can distribute 2 habitable planets $2!$ and 8 non-habitable planets $8!$ and there is 6 six different arrangements shown above. Total number of arrangement of planets around star is $ 6\cdot2!\cdot8!$ for one set.
\end{itemize}
We showed above we can choose $\frac{10!}{1!\cdot9!} \cdot \frac{20!}{2!\cdot18!} \cdot \frac{80!}{8!\cdot72!}$ different star and planet set and, for one set we have $6\cdot2!\cdot8!$ different arrangements. So by product rule our general result is :
$$\dfrac{10!}{1!\cdot9!} \cdot \dfrac{20!}{2!\cdot18!} \cdot \dfrac{80!}{8!\cdot72!} \cdot 6\cdot2!\cdot8!$$
\pagebreak
\section*{Answer 5}
\paragraph{a)} Let $S(n)$ is the number of ways to jump n cells. Robot can move 1st cell with 1 way. 2nd cell with 2 ways and 3rd cell with 4 ways.
\begin{center}
    \begin{tabular}{c|c|c}
        1 & 2 & 3\\
        \hline
        1 & 1+1 & 1+1+1 \\
        &2& 1+2 \\
        && 2+1\\
        && 3
    \end{tabular}
\end{center}
Robot can go nth cell by 
\begin{itemize}
    \item go (n-1)th cell and jump 1 cell
    \item go (n-2)th cell and jump 2 cell
    \item go (n-3)th cell and jump 3 cell
\end{itemize}
We can write this recurrence relation as follows for $ n >3 $: $$S(n) = S(n-1) + S(n-2) + S(n-3)$$
\paragraph{b)} Initial conditions are
\begin{itemize}
    \item $S(1) = 1$
    \item $S(2) = 2$
    \item $S(3) = 4$
\end{itemize}
\paragraph{c)}
$$S(4) = S(3) + S(2) + S(1) = 7$$
$$S(5) = S(4) + S(3) + S(2) = 13$$
$$S(6) = S(5) + S(4) + S(3) = 24$$
$$S(7) = S(6) + S(5) + S(4) = 44$$
$$S(8) = S(7) + S(6) + S(5) = 81$$
$$S(9) = S(8) + S(7) + S(6) = 149$$


\end{document}


